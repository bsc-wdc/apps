\section{complexsequence}
\label{module_complexsequence}
This module contains the following objects

\begin{itemize}
\item \ref{object_ComplexSequence} ComplexSequence

\item \ref{object_ComplexSequenceEvalSet} ComplexSequenceEvalSet

\end{itemize}
\subsection{Object ComplexSequence}

\label{object_ComplexSequence}

The ComplexSequence object has the following fields. To see how to access them refer to \ref{accessing_fields}
\begin{description}
\item{type} Type [int : Scalar] No documentation

\item{seq} Type [Sequence * : Scalar] No documentation

\item{data} Type [int * : Scalar] No documentation

\item{datastore} Type [int * : Scalar] No documentation

\item{depth} Type [int : Scalar] No documentation

\item{length} Type [int : Scalar] No documentation

\item{creator} Type [ComplexSequenceEvalSet * : Scalar]  what made it

\end{description}
A ComplexSequence is an abstraction of a 
Sequence which can be handily made using
ComplexSequenceEval functions and is efficiently
laid out in memory.




Member functions of ComplexSequence

\subsection{Object ComplexSequenceEvalSet}

\label{object_ComplexSequenceEvalSet}

The ComplexSequenceEvalSet object has the following fields. To see how to access them refer to \ref{accessing_fields}
\begin{description}
\item{type} Type [int : Scalar] No documentation

\item{has_been_prepared} Type [boolean : Scalar] No documentation

\item{left_window} Type [int : Scalar]  overall sequence eval 

\item{right_window} Type [int : Scalar]  overall sequence eval

\item{left_lookback} Type [int : Scalar]  overall sequence eval

\item{cse} Type [ComplexSequenceEval ** : List] No documentation

\end{description}
This object holds a collection of 
ComplexSequenceEval's. Its role is to
define the sequence specific parts of a
dynamic programming algorithm as computable
functions. 


Ideally you should use pre-made ComplexSequenceEvalSets
as it will save you alot of grief




Member functions of ComplexSequenceEvalSet

